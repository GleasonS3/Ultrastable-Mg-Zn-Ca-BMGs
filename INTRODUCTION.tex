\section{INTRODUCTION }

Current pharmaceutical technology relays on re-dosing of drugs, often with treatments being readminister several times per day or week. Coating pharmaceuticals with tailored bioabsorbable films designed to degrade over time could allow for a slow controlled release of drug packages such as antibiotics, antimicrobials, and analgesics (painkillers). These medical devices could be implanted during surgeries, eliminating the needs for daily drug administration.  

Thin film metallic glasses (\textit{TFMGs}) are one technology which may make this possible. These novel amorphous metallic materials been demonstrated to significantly modify substrate properties such as hardness, wear residence, surface finish, fatigue and corrosion residence, and even ductility. Recently ultrastable metallic glass (\textit{SMG}) films have been discovered which display improved thermal and kinetic stabilities, and often reduced entropy relative to the more established \textit{TFMGs}. Many properties of \textit{SMGs} have not yet been characterised and it is not yet known if in additional to improved stability if they may offer greater improvements in substrate property modification. The application of these films onto drug delivery systems, or even orthopaedic devices, could provide great potential for improvements in wound healing and pain management practices.

The aims of this thesis are to produce and investigate quality \textit{TFMGs} and \textit{SMGs} for biomedical applications. Thin films of established bulk metallic glass (\textit{BMG}) compositions, such as Mg$_{65}$Zn$_{30}$Ca$_{5}$ will be deposited onto various substrates including; \textit{BMGs} of similar film composition, Polycaprolactone (\textit{PCL}) scaffolds, and dissolvable NaCl substrate (to allow base film to be studied independently). The properties and characteristics of the films as well as their property modification effects on the different substrates will be investigated, and the characterised films compared with their \textit{BMG} counterparts. 

\begin{equation}
\nabla_{\rm rad} = \frac{3F\kappa}{4acg}\frac{P}{T^4} > \nabla_{\rm ad}.
\end{equation}
