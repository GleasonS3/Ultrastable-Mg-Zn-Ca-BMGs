\section{ABSTRACT}
 
The aims of this thesis are to produce and examine biodegradable thin film metallic glasses (\textit{TFMGs}) and the recently discovered ultrastable metallic glass (\textit{SMG}) films for biomedical applications. To ensure full biocompatibility the films will be composed entirely of essential mineral \textit{MgZnCa} alloys.

The literature review provides an overview of metallic glass formation and processing, thin films and deposition methods, initial understandings of ultrastable glasses, and biomedical requirements with a focus on biodegradation. 

The films will be deposited onto various substrates via magnetron sputtering and pulse laser deposition (\textit{PLD}) techniques. The master alloys and deposition targets will be prepared via induction furnace melting and copper mould gravity casting of pure base element Mg, Zn, and Ca (99.8wt\% pure or better).

The initial results have shown the Mg_{65}Zn_{30}Ca_{5} alloy is relatively brittle, with most targets fracturing during casting or shaping operations. The targets which were produced without failure were examined via \textit{DSC} and found to be at best primarily crystalline in structure. This should be expected as this alloy’s critical casting thickness is similar to the thickness of the mould utilised. 

Going forward \textit{XRD} analysis will be used to definitively establish the targets’ structures, and the target manufacturing process will be refined (current method is not efficient ). Numinous thin film metallic glass specimens will be produced via sputtering/\textit{PLD} and evaluated; hopefully leading to publications and attendance at conferences. 

Mg^{65}